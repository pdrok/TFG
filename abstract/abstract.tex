\thispagestyle{empty}
\begin{center}
\begin{LARGE}
\textbf{Abstract}
\end{LARGE}
\end{center}

\begin{quotation}
The present project implemented an alternative to improve the security of data on a medium-large enterprise, centralizing the servers through the implementation of a VPN by means of radio frequencies and backup link via Internet.\\
The advantages of this implementation were immediate, be it with the reduction of the costs in infrastructure, reduction of the administrative expenses and increase in the resilience of the service among others.\\
A connection via VPN allowed the centralization and best management of the data without many complications for its maintenance nor the need to buy expensive or complex equipment (traditional physical servers).\\
Centralizing the servers helped in managing them, reduced the costs and facilitate the data management; in addition, with the reduction of servers the backup take less time.\\
To implement this infrastructure it was made use of the RouterOS operating system based on GNU/Linux for its high reliability and its easy management through a graphical interface, as well as access via Telnet, SSH and Web, considering the aforementioned, the router Mikrotik is represented in the market as a strong solution, professional and economically feasible.
\vspace*{0.5cm}

\noindent {\bf Key words:} 1. VPN, 2. Mikrotik,, 3. RouterOS. 

\end{quotation}
