\thispagestyle{empty}
\begin{center}
\begin{LARGE}
\textbf{Resumen}
\end{LARGE}
\end{center}
\begin{quotation}

El presente proyecto, implement� una alternativa para mejorar la disponibilidad de los datos de una empresa de porte media a grande centralizando servidores, mediante la  implementaci�n de una VPN por medio de radio frecuencias y respaldo de enlace v�a Internet.\\
Los beneficios de esta implementaci�n fueron inmediatos, ya sea con la reducci�n de costos de infraestructura, reducci�n de los gastos administrativos, y aumento de la resiliencia del servicio, entre otros.\\
Una conexi�n por medio VPN, permiti� la centralizaci�n y el mejor manejo de la informaci�n, sin muchas dificultades para su mantenimiento, ni la necesidad de comprar equipamientos caros o complejos (servidores f�sicos tradicionales).\\
Centralizar los servidores ayud� en la administraci�n de los mismos, redujo los costos y facilit� el manejo de los datos; adem�s con la disminuci�n de la cantidad servidores el respaldo se realiza en menor tiempo.\\
Para implementar esta infraestructura se hizo uso del sistema operativo RouterOS basado en GNU/Linux por su alta confiabilidad y su f�cil administraci�n por medio de un entorno gr�fico, as� como accesos v�a Telnet, SSH y web; teniendo en cuenta lo citado, el enrutador Mikrotik utilizado en este trabajo se presenta en el mercado como una soluci�n robusta, profesional y econ�micamente factible en comparaci�n a otros enrutadores de otros fabricantes con similares prestaciones.
\vspace*{0.5cm}

\noindent {\bf Descriptores:} 1. VPN, 2. Mikrotik, 3. RouterOS.

\end{quotation}
