\fancyhead{}
\fancyfoot{}


\lhead{Conclusi�n y recomendaciones}

\chapter{Conclusi�n y recomendaciones}


El modelo de infraestructura que responde de manera m�s pr�ctica y eficiente para la centralizaci�n de datos es utilizando VPN a trav�s de radio enlaces y reforzando con enlaces de internet.
Los equipos hardware y software utilizados para la implementaci�n de la centralizaci�n de datos a trav�s de VPN responden de manera eficiente al costo beneficio, la marca Mikrotik es mucha m�s barata que otras tecnolog�as de su competencia como Cisco. Presenta la robustez necesaria para los radios enlaces.
La cantidad de recursos humanos necesarios y el orden de desarrollo del trabajo para la implementaci�n de la centralizaci�n de datos a trav�s de VPN fueron evaluados y demostraron ventajas en costos beneficios. Antes se necesitaba si o si una persona de TI por cada local, con la implementaci�n de la centralizaci�n ya no existe esa necesidad. Adem�s, se economiza los costos de la locomoci�n.
Las recomendaciones para tener una seguridad m�nima en la centralizaci�n de los datos a trav�s de VPN es la administraci�n eficiente de las mismas, ya que todos los equipos re�nen las condiciones necesarias para la implementaci�n de la seguridad.
Las 21 filiales anexadas a la central y la administraci�n de la mismas de manera remota genera una gran econom�a a la empresa y es un modelo replicable a otras empresas que buscan mejor infraestructura para la administraci�n de su informaci�n.
Se recomienda la instalaci�n de enlaces con equipamientos RouterBoard por la practicidad, robustez y econom�a.
