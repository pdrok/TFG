\newacronym{vpn}{VPN}{Virtual Private Network}
\glsreset{vpn} 
%\newacronym{ti}{TI}{Técnologia de la Información}
%\glsreset{ti} 

\newglossaryentry{servidor}
{
name=Servidor,
description={Un servidor es una aplicación en ejecución (software) capaz de atender las peticiones de un cliente y devolverle una respuesta en concordancia. Los servidores se pueden ejecutar en cualquier tipo de computadora, incluso en computadoras dedicadas a las cuales se les conoce individualmente como «el servidor». En la mayoría de los casos una misma computadora puede proveer múltiples servicios y tener varios servidores en funcionamiento. La ventaja de montar un servidor en computadoras dedicadas es la seguridad. Por esta razón la mayoría de los servidores son procesos diseñados de forma que puedan funcionar en computadoras de propósito específico.},
plural={Servidores},
}

 \newglossaryentry{traceroute}
{
name=Traceroute,
description={Traceroute es una consola de diagnóstico que permite seguir la pista de los paquetes que vienen desde un host (punto de red)}
}

\newglossaryentry{rf}
{
name=RF,
description={también denominado espectro de radiofrecuencia, se aplica a la porción menos energética del espectro electromagnético, situada entre 3 hercios (Hz) y 300 gigahercios (GHz)}
}




\newacronym{nas}{NAS}{Network Access Server}
\glsreset{nas} 

\newglossaryentry{CCIR}
{
  name=CCIR,
  description={Comité Consultivo Internacional de Radiocomunicaciones (CCIR), o International Radio Consultative Committee (IRCC), con el objeto de servir como un comité de normalización de las radiocomunicaciones.},
}
%\newacronym{am}{AM}{Sistema de modulación de amplitud}
%\glsreset{am} 
\newglossaryentry{am}
{
name=AM,
description={La modulación de amplitud o amplitud modulada (AM) es una técnica utilizada en la comunicación electrónica, más comúnmente para la transmisión de información a través de una onda transversal de televisión. La modulación en amplitud (AM) funciona mediante la variación de la amplitud de la señal transmitida en relación con la información que se envía.}
}

%\newacronym{fm}{FM}{Sistema de modulación de frecuencia}
%\glsreset{fm} 

\newglossaryentry{fm}
{
name=FM,
description={La modulación de frecuencia,o frecuencia modulada (FM), es una técnica de modulación que permite transmitir información a través de una onda portadora variando su frecuencia. En aplicaciones analógicas, la frecuencia instantánea de la señal modulada es proporcional al valor instantáneo de la señal moduladora}
}
\newglossaryentry{hf}
{
name=HF,
description={Alta frecuencia}
}

%\newacronym{v/m}{V/m}{diferencia de potencial por metro}
%\glsreset{v/m}
\newglossaryentry{v/m}
{
name=V/m,
description={Un campo eléctrico es un campo de fuerza creado por la atracción y repulsión de cargas eléctricas (la causa del flujo eléctrico) y se mide en Voltios por metro (V/m). El flujo decrece con la distancia a la fuente que provoca el campo.}
}
%\newacronym{db}{dB}{Decibeles}
%\glsreset{db}
\newglossaryentry{db}
{
name=dB,
description={Expresa una razón entre cantidades y no una cantidad. El decibel expresa cuantas veces más o cuantas veces menos, pero no la cantidad exacta. Es una expresión que no es lineal, sino logarítmica. Es una unidad de medida relativa. En audiofrecuencias un cambio de 1 decibel (dB) es apenas (si hay suerte) notado.}
}
\newacronym{WAN}{WAN}{Wide Area Network}
\glsreset{WAN} 

%\newacronym{ssl/tls}{SSL/TLS}{Secure Sockets Layer/Transport Layer Security}
%\glsreset{ssl/tls} 

\newglossaryentry{ssl/tls}
{
name=SSL/TLS,
description={Transport Layer Security (TLS; en español seguridad de la capa de transporte) y su antecesor Secure Sockets Layer (SSL; en español capa de conexión segura) son protocolos criptográficos que proporcionan comunicaciones seguras por una red, comúnmente}
}
\newacronym{ipsec}{IPsec}{Internet Protocol security}
\glsreset{ipsec} 

\newacronym{gre}{GRE}{Generic Routing Encapsulation}
\glsreset{gre} 

\newacronym{vpws}{VPWS}{Virtual Private Wire Service}
\glsreset{vpws} 

\newacronym{vpls}{VPLS}{Virtual Private Lan Service}
\glsreset{vpls} 

\newacronym{ipls}{IPLS}{IP only Private Lan Service}
\glsreset{ipls} 

\newacronym{mpls}{MPLS}{Multiprotocol Label Switching}
\glsreset{mpls} 

\newacronym{atom}{AtoM}{Any Transport over MPLS}
\glsreset{atom} 
\newacronym{l2tpv3}{L2TPv3}{Layer 2 Tunneling Protocol version 3}
\glsreset{l2tpv3} 
\newacronym{mplslsp}{MPLSLSP}{MPLS Label Switched Path}
\glsreset{mplslsp} 
\newglossaryentry{ieee802q}
{
name=IEEE 802.1Q,
description={El protocolo IEEE 802.1Q, también conocido como dot1Q, fue un proyecto del grupo de trabajo 802 de la IEEE para desarrollar un mecanismo que permita a múltiples redes compartir de forma transparente el mismo medio físico, sin problemas de interferencia entre ellas (Trunking).}
}
\newacronym{pptp}{PPTP}{Point to Point Tunneling Protocol}
\glsreset{mplslsp} 

\newacronym{l2f}{L2F}{Layer 2 Forwarding}
\glsreset{l2f} 

\newacronym{mac}{MAC}{Media Access Control}
\glsreset{mac} 

\newacronym{dlci}{DLCI}{Frame Relay Data Link Connection Identifier}
\glsreset{dlci} 

\newacronym{m2m}{M2M}{Multipunto a Multipunto}
\glsreset{m2m} 

\newacronym{p2p}{P2P}{Punto a Punto}
\glsreset{p2p} 

\newacronym{ti}{TI}{Técnologia de la Información}
\glsreset{ti} 
\newacronym{bgp}{BGP}{Border Gateway Protocol}
\glsreset{bgp} 
\newglossaryentry{atm}
{
name=ATM,
description={El modo de transferencia asíncrona (Asynchronous Transfer Mode, ATM) es una tecnología de telecomunicación desarrollada para hacer frente a la gran demanda de capacidad de transmisión para servicios y aplicaciones, basada en la conmutación por etiquetas.}
}

\newacronym{https}{HTTPS}{Hypertext Transfer Protocol Secure}
\glsreset{https} 

\newacronym{qos}{QoS}{Quality of Service}
\glsreset{qos} 

\newacronym{sla}{SLA}{Service Level Agreement}
\glsreset{sla} 
\newacronym{slo}{SLO}{Service Level Objective}
\glsreset{slo} 
\newacronym{sls}{SLS}{Service Level Specification}
\glsreset{sls} 
\newacronym{ssl}{SSL}{Secure Sockets Layer}
\glsreset{ssl} 
\newglossaryentry{ssh}
{
name=SSH,
description={ protocolo y del programa que lo implementa, y sirve para acceder a máquinas remotas a través de una red. Permite manejar por completo la computadora mediante un intérprete de comandos, y también puede redirigir el tráfico de X (Sistema de Ventanas X) para poder ejecutar programas gráficos si tenemos ejecutando un Servidor X (en sistemas Unix y Windows).}
}

%\newacronym{ftp}{FTP}{File Transfer Protocol}
%\glsreset{ftp} 

\newglossaryentry{ftp}
{
name=FTP,
description={protocolo de red para la transferencia de archivos entre sistemas conectados a una red TCP (Transmission Control Protocol), basado en la arquitectura cliente-servidor. Desde un equipo cliente se puede conectar a un servidor para descargar archivos desde él o para enviarle archivos, independientemente del sistema operativo utilizado en cada equipo.}
}
\newglossaryentry{telnet}
{
name=Telnet,
description={Protocolo de red para la transferencia de archivos entre sistemas conectados a una red TCP (Transmission Control Protocol), basado en la arquitectura cliente-servidor. Desde un equipo cliente se puede conectar a un servidor para descargar archivos desde él o para enviarle archivos, independientemente del sistema operativo utilizado en cada equipo.}
}
%\newacronym{telnet}{Telnet}{Telecommunication Network}
%\glsreset{telnet} 

\newacronym{itm}{ITM}{Irregular Terrain Model}
\glsreset{itm} 
\newacronym{strm}{STRM}{Shuttle Terrain Radar Mapping Misión}
\glsreset{strm} 
\newacronym{kml}{.kml}{Keyhole Markup Language}
\glsreset{kml} 



\newglossaryentry{ssid}
{
name=SSID,
description={Service Set Identifier: es un nombre incluido en todos los paquetes de una red inalámbrica para identificarlos como parte de esa red}
}
\newacronym{ap}{AP}{Access Point}
\glsreset{ap} 
%\newacronym{mtu}{MTU}{Unidad máxima de transferencia}
%\glsreset{mtu} 

\newglossaryentry{mtu}
{
name=MTU,
description={es un término de redes de computadoras que expresa el tamaño en bytes de la unidad de datos más grande que puede enviarse usando un protocolo de comunicaciones.}
}

\newglossaryentry{mhz}
{
name=MHz,
description={Megahercio}
}
\newglossaryentry{khz}
{
name=KHz,
description={Kilohercio}
}
